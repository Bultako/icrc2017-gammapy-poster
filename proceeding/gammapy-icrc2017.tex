\documentclass{PoS}

\newcommand{\url}[1]{\href{#1}{#1}}

% Shortcuts
\newcommand{\aap}{AAP}
\newcommand{\pasp}{PASP}
\newcommand{\procspie}{SPIE}

\newcommand{\urlGammapySlack}{\href{https://gammapy.slack.com}{gammapy.slack.com}}
\newcommand{\urlCtaPipe}{\href{https://github.com/cta-observatory/ctapipe}{github.com/cta-observatory/ctapipe}}
\newcommand{\urlPint}{\href{https://github.com/nanograv/PINT}{github.com/nanograv/PINT}}
\newcommand{\urlFermipy}{\href{https://github.com/fermipy/fermipy}{github.com/fermipy/fermipy}}
\newcommand{\urlGammapyDocs}{\href{http://docs.gammapy.org}{docs.gammapy.org}}
\newcommand{\urlErfa}{\href{https://github.com/liberfa/erfa}{github.com/liberfa/erfa}}
\newcommand{\urlSofa}{\href{http://www.iausofa.org}{www.iausofa.org}}
\newcommand{\urlGammapyGithub}{\href{https://github.com/gammapy/gammapy}{github.com/gammapy/gammapy}}
\newcommand{\urlPytest}{\href{https://pytest.org}{pytest.org}}
\newcommand{\urlSphinx}{\href{http://www.sphinx-doc.org}{www.sphinx-doc.org}}
\newcommand{\urlJupyter}{\href{https://jupyter.org}{jupyter.org}}
\newcommand{\urlPypi}{\href{https://pypi.python.org}{pypi.python.org}}
\newcommand{\urlPip}{\href{https://pip.pypa.io}{pip.pypa.io}}
\newcommand{\urlAnacondaGammapy}{\href{https://anaconda.org/astropy/gammapy}{anaconda.org/astropy/gammapy}}
\newcommand{\urlGentooGammapy}{\href{https://packages.gentoo.org/packages/dev-python/gammapy}{packages.gentoo.org/packages/dev-python/gammapy}}
\newcommand{\urlGammapyForum}{\href{https://groups.google.com/forum/\#!forum/gammapy}{groups.google.com/forum/\#!forum/gammapy}}
\newcommand{\urlCtaAck}{\href{http://www.cta-observatory.org/consortium_acknowledgments}{www.cta-observatory.org/consortium\_acknowledgments}}
\newcommand{\urlGithub}{\href{https://github.com}{github.com}}
\newcommand{\urlRtd}{\href{https://readthedocs.org}{readthedocs.org}}
\newcommand{\urlTravis}{\href{https://travis-ci.org}{travis-ci.org}}
\newcommand{\urlAppveyor}{\href{https://appveyor.com}{appveyor.com}}
\newcommand{\urlSlack}{\href{https://slack.com}{slack.com}}

\title{Gammapy -- A prototype for the CTA science tools}
\ShortTitle{Gammapy -- A prototype for the CTA science tools}

\author{
Christoph Deil$^a$,
Roberta Zanin$^a$,
Julien Lefaucheur$^b$,
Catherine Boisson$^b$,
Bruno Kh\'elifi$^c$,
R\'egis Terrier$^c$,
Matthew Wood$^d$,
Lars Mohrmann$^e$,
Nachiketa Chakraborty$^a$,
Jason Watson$^a$,
Rub\'en L\'opez Coto$^a$,
Stefan Klepser$^f$,
\speaker{Matteo Cerruti}$^g$,
Jean-Philippe Lenain$^g$,
Fabio Acero$^h$,
Arache Djannati-Ata{\"\i}$^c$,
Santiago Pita$^c$,
Zeljka Bosnjak$^i$,
Jose Enrique Ruiz$^j$,
Cyril Trichard$^k$,
Thomas Vuillaume$^l$,
for the CTA Consortium,
Axel Donath$^a$,
Johannes King$^a$,
L\'ea Jouvin$^c$,
Ellis Owen$^m$,
Manuel Paz Arribas$^n$,
Brigitta Sipocz$^o$,
Dirk Lennarz$^p$,
Arjun Voruganti$^a$,
Marion Spir-Jacob$^c$
\\
\llap{$^a$}MPIK, Heidelberg, Germany\\
\llap{$^b$}LUTH, Obs. de Paris/Meudon, France\\
\llap{$^c$}APC/CNRS, Paris, France\\
\llap{$^d$}SLAC National Accelerator Laboratory, US\\
\llap{$^e$}FAU, Erlangen, Germany\\
\llap{$^f$}DESY, Zeuthen, Germany\\
\llap{$^g$}LPNHE, Paris, France\\
\llap{$^h$}CEA/IRFU, Saclay, France\\
\llap{$^i$}University of Rijeka, Croatia\\
\llap{$^j$}Instituto Astrof\'isica de Andaluc\'ia, Granada, Spain\\
\llap{$^k$}CPPM, Marseille, France\\
\llap{$^l$}LAPP, Annecy-le-Vieux, France\\
\llap{$^m$}UCL-MSSL, Dorking, United Kingdom\\
\llap{$^n$}Humboldt University, Berlin, Germany\\
\llap{$^o$}Cambridge, UK\\
\llap{$^p$}Georgia Tech, Atlanta, US\\
E-mail:
\email{Christoph.Deil@mpi-hd.mpg.de},
\email{Roberta.Zanin@mpi-hd.mpg.de},
\email{julien.lefaucheur@obspm.fr},
\email{catherine.boisson@obspm.fr},
\email{khelifi@apc.in2p3.fr},
}

\abstract{

Gammapy is a Python package for high-level gamma-ray data analysis built on
Numpy, Scipy and Astropy. Starting with event lists and instrument response
information, it is possible to analyse gamma-ray data and to create for example
sky images, spectra and lightcurves, and to determine the position, morphology
and spectra of gamma-ray sources.

So far Gammapy has mostly been used to analyse data from H.E.S.S. and Fermi-LAT,
and now it is being used for the simulation and analysis of observations from
the Cherenkov Telescope Array (CTA). We have proposed Gammapy as a prototype for
the CTA science tools. This contribution will give an overview of the Gammapy
package and show an analysis application example with simulated CTA data.

}

\FullConference{35th International Cosmic Ray Conference --- ICRC2017\\
		10--20 July, 2017\\
		Bexco, Busan, Korea}


\begin{document}

\section{Introduction}
\label{sec:intro}

% CTA and CTA science tools introduction
The Cherenkov Telescope Array (CTA) will observe the sky in very-high-energy
(VHE, E > 20$\,$GeV) gamma-ray light soon. CTA will consist of large telescope
arrays at two sites, one in the southern (Chile) and one in the northern (La
Palma) hemisphere. It will perform surveys of large parts of the sky, targeted
observations on Galactic and extra-galactic sources, and more specialized
analyses like a measurement of charged cosmic rays, constraints on the
intergalactic medium opacity for gamma-rays and a search for dark matter.
Compared to current Cherenkov telescope arrays such as H.E.S.S., VERITAS or
MAGIC, CTA will have a much improved detection area, angular and energy
resolution, improved signal/background classification and sensitivity. CTA is
expected to operate for thirty years, and all astronomers will have access to
CTA high-level data, as well as CTA science tools (ST) software. The ST can be
used for example to generate sky images and to measure source properties such as
morphology, spectra and light curves, using event lists as well as instrument
response function (IRF) and auxiliary information as input.

% What is Gammapy?
Gammapy is a prototype for the CTA ST, built on the scientific Python stack and
Astropy \cite{astropy}, optionally using Sherpa \cite{sherpa2001, sherpa2009,
sherpa2011} or other packages for modeling and fitting (see
Figure~\ref{fig:stack}). Initially, the focus was to implement the ``classical
TeV analysis'', using 2-dimensional sky images for source detection and
morphology fitting, followed by spectral analysis for a given source region. A
3\hbox{-}dimensional analysis with a simultaneous spatial and spectral model of
the gamma-ray emission, as well as background (called ``cube analysis'' in the
following) is in development.
% Existing studies using Gammapy
A first study comparing spectra obtained with the classical 1D analysis and the
3D cube analysis using point source observations with H.E.S.S., to compare the
methods and to validate Gammapy, is presented in \cite{lea}. Further
developments and verification using data from existing Cherenkov telescope
arrays such as H.E.S.S. and MAGIC, as well as simulated CTA data is ongoing.
% Gammapy use in CTA
Gammapy is starting to be used for scientific studies for existing ground-based
gamma-ray telescopes \cite{hgps, shells}, the Fermi-LAT space telescope
\cite{owen2015}, as well as for CTA \cite{julien, roberta, cyril}.

\begin{figure}[t]
\centering
\includegraphics[width=0.7\textwidth]{figures/gammapy-stack}
\caption{
The Gammapy stack. Required dependencies Numpy and Astropy are illustrated with
solid arrows, optional dependencies (the rest) with dashed arrows.
}
\label{fig:stack} \end{figure}

% Outline of this paper
In this writeup we focus on the software and technical aspects of Gammapy. We
start with a brief overview of the context in Section~\ref{sec:context},
followed by a description of the Gammapy package in Section~\ref{sec:package},
the Gammapy project in Section~\ref{sec:project} and finally our conclusions
concerning Gammapy as as CTA science tool prototype in
Section~\ref{sec:conclusions}.

\section{Context}
\label{sec:context}

% Python, Numpy, Astropy
Before moving on to a description of Gammapy in the next section, we would like
to give some context for its development and mention some related projects. An
early prototype package that was similar to Gammapy was ``PyFACT: Python and
FITS Analysis for Cherenkov Telescopes'' \cite{pyfact}. It was developed in
2011/2012 and hasn't been updated since, we mainly mention it as a first  idea
to build the CTA science tools as a Python package using Numpy. In 2011, the
Astronomical Python community came together and created the Astropy project and
package \cite{astropy}, which is a key factor making Python the most popular
language for astronomical research codes (at least according to this informal
survey \cite{momcheva2015}). Gammapy is an Astropy affiliated package, which
means that where possible it uses the Astropy core package instead of
duplicating its functionality, as well as having a certain quality standard such
as having automated tests and documentation for the available functionality. In
recent years, several other packages have adopted the same approach, to build on
Python, Numpy and Astropy. To name just a few, there is ctapipe (\urlCtaPipe),
the prototype for the low-level CTA data processing pipeline (up to DL3); Naima
for modeling the non-thermal spectral energy distribution of astrophysical
sources \cite{naima}; and PINT, a new software for high-precision pulsar timing
(\urlPint), and Fermipy (\urlFermipy), a Python package that facilitates and
analysis of data from the Large Area Telescope (LAT) with the Fermi Science
Tools and adds some extra functionality.

We note that many other astronomy projects have chosen Python/Astropy as the
basis both for their data calibration and reduction pipeline as well as the
science tools used by astronomers. Some prominent examples are the Hubble space
telescope (HST) (TODO: add reference), the upcoming James Webb Space Telescope
(JWST) (TODO: add reference) and the Chandra X-ray observatory \cite{chandra,
sherpa2001}. Even projects like LSST that started their analysis software
developments before Astropy existed and are based on C++/SWIG are now actively
looking for ways to collaborate and make their software stack interoperate with
Numpy and Astropy to avoid code duplication, but also to leverage the fact that
a large fraction of the astronomical community already knows and is using
Astropy \cite{lsst}.

% Comparison to Gammalib, ctools
Another open-source package that has been proposed as a prototype for the CTA
science tools is Gammalib/ctools \cite{ctools}. Gammalib is a C++ library with
SWIG Python wrappers that doesn't have any dependencies besides CFITSIO, instead
implementing the functionality needed from scratch. ctools is a set of command
line tools, following the FTOOLs model of using FITS files as input and output
and being able to chain them into analysis chains using Python. Concerning
analysis methods, Gammalib/ctools is supporting binned and unbinned likelihood
analysis and implemented ``cube analysis'' first, adding support for ``classical
analysis'' now (the other way around compared to Gammapy).

% Open data formats
The choice for the official CTA science tools, supported and distributed by the
CTA observatory, has not been made yet; at this time both Gammapy and
Gammalib/ctools are open-source codes being used for CTA as well as exiting
gamma-ray telescopes. One issue noticed in the past years was that in the
current situation of having multiple telescopes converting their data to FITS
format and multiple science tools, it became necessary to write down the details
of the data formats being used. At this time, the data formats used in existing
science tool codes (Gammapy, Gammalib/ctools and partly also others, like
Fermipy, 3ML \cite{3ml} or Naima \cite{naima}) are to a large degree the same,
although one has to mention that the development of the DL3 data model and
formats is work in progress, and especially the IRF formats and DL3 data linking
of events to IRFs to support multiple event types will have to be extended
\cite{opendata}, a process Gammapy is participating in and contributing to.

\section{Gammapy package}
\label{sec:package}

Gammapy is a Python package built on Numpy \cite{numpy}, Scipy \cite{scipy} and
Astropy \cite{astropy}. Optional dependencies are Scipy for integration and
interpolation, and Sherpa \cite{sherpa2001, sherpa2009, sherpa2011} for modeling
and fitting and Matplotlib \cite{matplotlib} for plotting. The Gammapy
dependency stack is shown in Figure~\ref{fig:stack}.

The functionality is organized into sub-packages, such as for example {\it
gammapy.data},\linebreak{\it gammapy.irf}, {\it gammapy.spectrum}, \ldots . The
Gammapy features are described in detail in the Gamma\-py documentation
(\urlGammapyDocs) and many examples given in the tutorial-style Jupyter
notebooks, as well as in \cite{gammapy-icrc2015}. Here, we wanted to focus on
Gammapy as a software package and a new approach to build the CTA science tools.
So instead of listing the currently available functionality here, we will
continue in the next section with a code example explaining how Gammapy works.

% \section{How Gammapy works: a code example}
% \label{sec:code}

Gammapy is written high-level Python code, the data is stored in Numpy arrays or
objects such as {\it astropy.coordinates.SkyCoord} or {\it astropy.table.Table}
that hold Numpy array data members. Almost all functionality needed has been
written in C and Python wrappers already. Specifically, {\it astropy.wcs} is
calling into WCSLib (\cite{wcslib}), {\it astropy.io.fits} uses CFITSIO
(\cite{cfitsio}) and {\it astropy.coordinates} as well as {\it astropy.time} are
built on ERFA (\urlErfa), the open-source variant of the IAU Standards of
Fundamental Astronomy (SOFA) C library\linebreak (\urlSofa).

An example script that generates a counts image from an event list using Gammapy
is shown in Figure~\ref{fig:code_example}. The point we want to
make here is that it is possible to efficiently work with events and pixels and
to implement algorithms from Python, by storing all data in Numpy arrays and
processing via calls into existing C extensions in Numpy and Astropy. E.g. here
{\it EventList} stores the RA and DEC columns from the event list as Numpy
arrays, and {\it SkyImage} the pixel data as well, and {\it image.fill(events)},
and all processing happens in existing C extensions (Numpy histograms and
Astropy calls into the CFITSIO and WCSLib C libraries). Should the need arise,
that some algorithm can't be efficiently implemented in pure Python, a small C
extension can be added to Gammapy, using e.g. Cython \cite{cython}.

We note that a huge eco-system of scientific Python packages that operate on
Numpy arrays is available for advanced users, to implement analysis methods for
special use cases (e.g. sky or background models, IRF handling or likelihood or
Bayesian analysis methods). To name just a few that we have seen used in
conjunction with Gammapy so far (by users, not within the Gammapy package
itself): healpy, Scipy, pandas, scikit-image, scikit-learn, iminuit, emcee.

\begin{figure}[t]
\centering
\includegraphics[width=0.5\textwidth]{examples/code_events_image}
\caption{
An example script using Gammapy to make a counts image from an event list. This
is used in Section~\ref{sec:package} to explain how Gammapy achieves efficient
processing of event and pixel data from Python: all data is stored in Numpy
arrays and passed to existing C extensions in Numpy and Astropy.
}
\label{fig:code_example}
\end{figure}

% \section{Application example}
% \label{sec:application}

In this poster we focused on the software and technical aspects of Gammapy. For
examples of CTA science studies using Gammapy, we refer you to other posters
presented at this conference: Galactic survey \cite{roberta}, PeVatrons
\cite{cyril} and extra-galactic sources \cite{julien}.

Several other examples using real data from H.E.S.S. and Fermi-LAT, as well as
simulated data for CTA can be found via \urlGammapyDocs\ by following the link
to ``tutorial notebooks''. Figure~\ref{fig:app} shows one result of the ``CTA
data analysis with Gammapy'' notebook: a significance sky image of the Galactic
center region using 1.5~hours of simulated CTA data. The background was
estimated using the ring background estimation technique, and peaks above
5~sigma are shows with white circles.

\begin{figure}[t]
\centering
\includegraphics[width=0.7\textwidth]{figures/gammapy_example_sky_image.png}
\caption{
Application example: significance image for the Galactic centre region using
1.5~hours of simulated CTA data.  White circles are peaks above 5~sigma.
}
\label{fig:app}
\end{figure}

\section{Gammapy project}
\label{sec:project}

In this section we describe the current setup of the Gammapy project. We are using the common tools and services for Python open-source projects for software
development, code review, testing, documentation, package distribution and user
support.

Gammapy development happens on Github (\urlGammapyGithub). We make extensive use
of the pull request system to discuss and review code contributions. For testing
we use pytest (\urlPytest), for continuous integration Travis-CI (Linux and Mac)
as well as Appveyor (Windows). For documentation Sphinx (\urlSphinx), for
tutorial-style documentation Jupyter notebooks (\urlJupyter) are used.

Gammapy is distributed and installed in the usual way for Python packages. Each
stable release is uploaded to the Python package index (\urlPypi), and
downloaded and installed by users via {\it pip install gammapy} (\urlPip).
Binary packages for conda are available via the conda Astropy channel
(\urlAnacondaGammapy) for Linux, Mac and Windows, which conda users can install
via {\it conda install gammapy -c astropy}. Binary packages for the Macports
package manager are also available, which users can install via {\it port
install gammapy}. At this time, Gammapy is also available as a Gentoo Linux
package (\urlGentooGammapy) and a Debian Linux package is in preparation.

For Gammapy developer team communication we use Slack (\urlGammapySlack). A
public mailing list for user support and discussion is available
(\urlGammapyForum). Two face-to-face meetings for Gammapy were organized so far,
the first on in June 2016 in Heidelberg as a coding sprint for developers only,
the second on in February 2017 in Paris as a workshop for both Gammapy users and
developers.

% Note that the current setup described in this section is different from what it
% will be if Gammapy is chosen to be the or part of the CTA science tools, since
% CTA will set up their own systems for software development, maintenance,
% testing, documentation, issue tracking, software distribution and user support.

\section{Conclusions}
\label{sec:conclusions}

In the past two years, we have developed Gammapy as an open-source analysis
package for existing gamma-ray telescope and as a prototype for the CTA science
tools. Gammapy is a Python package, consisting of functions and classes that can
be used as a flexible and extensible toolbox to implement and execute high-level
gamma-ray data analyses.

We find that the Gammapy approach, to build on the powerful and well-tested
Python packages Numpy and Astropy, brings large benefits: A small codebase that
is focused on gamma-ray astronomy in a single high-level language is easy to
understand and maintain. It is also easy to modify and extend as new use cases
arise, which is important for CTA, since it can be expected that the modeling of
the instrument, background and astrophysical emission, as well as the analysis
method in general (e.g. likelihood or Bayesian statistical methods) will evolve
and improve over the next decade. Last but not least, the Gammapy approach is
inherently collaborative, sharing development effort as well as know-how with
the larger astronomical community, that to a large degree already has adopted
Numpy and Astropy as the basis for astronomical analysis codes in the past 5
years.

\section{Acknowledgements}
\label{sed:acknowledgements}

This work was conducted in the context of the CTA Consortium. We gratefully
acknowledge financial support from the agencies and organizations listed here:\\
\urlCtaAck

We would like to thank the Scientific Python and specifically the Astropy
community for providing their packages which are invaluable to the development
of Gammapy, as well as tools and help with package setup and continuous
integration, as well as building of conda packages.

We thank the GitHub (\urlGithub) team for providing us with an excellent free
development platform, ReadTheDocs (\urlRtd) for free documentation hosting,
Travis (\urlTravis) and Appveyor (\urlAppveyor) for free continuous integration
testing, and Slack (\urlSlack) for a free team communication channel.

\bibliography{gammapy-icrc2017}
\bibliographystyle{JHEP}

\end{document}
